%----------------------------------------------------------------------------------------
%	PACKAGES AND OTHER DOCUMENT CONFIGURATIONS
%----------------------------------------------------------------------------------------


\documentclass[a4paper,12pt]{article}
\usepackage[english]{babel}
\usepackage[latin1]{inputenc}
\usepackage{amsmath}
\usepackage{amssymb}
\usepackage{amsfonts}
\usepackage{graphicx}
\usepackage[colorinlistoftodos]{todonotes}
\usepackage[toc,page]{appendix}
\usepackage{setspace}
\doublespacing
\usepackage{booktabs}
\usepackage{geometry}
\usepackage[bottom]{footmisc}
\usepackage{longtable}
% \usepackage[demo]{graphicx}
\usepackage{subfig}
\usepackage{multirow}
\usepackage{tikz}
\usetikzlibrary{fit}
\usetikzlibrary{arrows}
\renewcommand{\arraystretch}{0.7}
\renewcommand{\labelitemi}{$\triangleright$}
 \geometry{
 a4paper,
 total={170mm,257mm},
 left=25mm,
 top=30mm,
 right=25mm,
 bottom=25mm,
 }
 \usepackage{hyperref}
 \hypersetup{
    bookmarks=true,         % show bookmarks bar?
    unicode=false,          % non-Latin characters in Acrobat’s bookmarks
    pdftoolbar=true,        % show Acrobat’s toolbar?
    pdfmenubar=true,        % show Acrobat’s menu?
    pdffitwindow=false,     % window fit to page when opened
    pdfstartview={FitH},    % fits the width of the page to the window
    pdftitle={Design_Report},    % title
    pdfauthor={Jacob Pichelman, Luca Poll},     % author
    pdfsubject={Subject},   % subject of the document
    pdfcreator={Creator},   % creator of the document
    pdfproducer={Producer}, % producer of the document
    pdfkeywords={keyword1, key2, key3}, % list of keywords
    pdfnewwindow=true,      % links in new PDF window
    colorlinks=false,       % false: boxed links; true: colored links
    linkcolor=red,          % color of internal links (change box color with linkbordercolor)
    citecolor=green,        % color of links to bibliography
    filecolor=magenta,      % color of file links
    urlcolor=cyan           % color of external links
}

	% biblatex
\usepackage[
       style=authoryear,
       natbib=true,
       maxcitenames=3,
       maxbibnames=11,
       backend=biber,
       pagetracker=page,
       hyperref=true,
       doi=true, 
       mergedate=compact, 
       firstinits=true
    ]{biblatex} 
	\usepackage{csquotes}
	\renewcommand*{\bibsetup}{
		\interlinepenalty=10000\relax % default is 5000
		\widowpenalty=10000\relax
		\clubpenalty=10000\relax
		\raggedbottom
		\frenchspacing
        \biburlsetup}
    \addbibresource{library.bib}

\begin{document}
\begin{titlepage}

\newcommand{\HRule}{\rule{\linewidth}{0.25mm}} % Defines a new command for the horizontal lines, change thickness here
\setlength{\topmargin}{-0.5in}
\center % Center everything on the page

\includegraphics[scale=0.75]{TSE.png}\\

%----------------------------------------------------------------------------------------
%	HEADING SECTIONS
%----------------------------------------------------------------------------------------
% \\[1.5cm]
\large \textsc{M2 EEE Panel Data} 
\vspace{1.5cm}
% Name of your heading such as course name
\textsc{\large } % Minor heading such as course title

%----------------------------------------------------------------------------------------
%	TITLE SECTION
%----------------------------------------------------------------------------------------

\HRule \\[0.75cm]
{ \huge \bfseries Panel Data Replication Project}\\[0.5cm] % Title of your document
\HRule \\[1.75cm]
 
%----------------------------------------------------------------------------------------
%	AUTHOR SECTION
%----------------------------------------------------------------------------------------

\large\textsc{Andrew Boomer, \\ Jacob Pichelmann, \\Luca Poll} \\[1.5cm]

%----------------------------------------------------------------------------------------
%	DATE SECTION
%----------------------------------------------------------------------------------------

{\large \today}\\[0.5cm] % Date, change the \today to a set date if you want to be precise

\vfill % Fill the rest of the page with whitespace

\end{titlepage}

%-------------------------------------------------------------
% TABLE OF CONTENTS
\renewcommand{\contentsname}{Table of Contents}
\tableofcontents
\clearpage
%-----------------------------------------------------------------------------

\section{Introduction}
After the Arab spring and the related outbreak of unforeseen violence, conflict forecasting models were largely criticized, and it was argued that forecasting new civil wars might have reached a limit. Mueller and Rauh (2018) though show in their paper "Reading between the lines: Prediction of political violence", that this might not be entirely true. Their main argument is structured as follows: Conventional conflict forecasting models\footnote{\noindent They demonstrate their argument by replicating the following papers on conflict prediction: \begin{itemize}
    \item Miguel \& Satyanath (2011): Prediction through rainfall growth
    \item Besley \& Presson (2011): Prediction through proxies for external shocks and political constraints
    \item Goldstone et al. (2010): Prediction through political institution dummies, child mortality rates, share of population discriminated against and whether neighboring countries in conflict
    \item Ward et al. (2013): Event database on high-intensity and low-intensity conflict events used for analysis
    \item Chadefaux (2014): Conflict prediction through analysis of keyword count in newspaper text
\end{itemize}}, that rely on the overall variation in country fixed effect models, exhibit a bias towards predicting conflict onset to where conflict has occurred before. This is partially due to large country fixed effects and slow moving factors like population, ethnic fractionalization, climate, etc. that result in a large between variation. The forecasts are hence dominated by structural time-invariant (or slow moving) factors, neglecting valuable within variation. As a result these models are relatively good at predicting (biasedly) where conflict will happen, but not when it will happen. In order to improve the forecasting of the timing of conflict and generate an unbiased forecast, Mueller \& Rauh (2018) propose to isolate the within from the overall variation and use such to predict the onset of armed conflict and civil war. In order to obtain necessary within variation, they propose using topic modeling on newspaper text to create variables of the average distribution of topic shares observed in a country during a given year. 



\section{Sample \& Data}

The sample for the underlying empirical analysis consists of 700.000 newspaper articles from three internationally-reporting newspapers between 1975 and 2015: the Economist\footnote{174.450 articles from 1975 onward}, the New York Times\footnote{363.275 articles from 1980 onward} and the Washington Post\footnote{185.523 articles from 1977 onward}. The newspapers cover in total 185 countries and the average yearly coverage amounts to 120 articles per country (with a range from 1 to 5.500). The authors use an unsupervised learning algorithm to break these articles into 15 distinct topic groups. \\

The dependent variables on the other hand are constructed through battle-related deaths from the Uppsala Conflict Data Program (UCDP/PRIO). Following their definition, armed conflict (dep. var. 1) is defined as a contested incompatibility that concerns government and/or territory over which the use of armed force between two parties, of which at least one is the government of a state, has resulted in at least 25 battle-related deaths in one calendar year. Civil conflict (dep. var. 2) follows the same definition but requires at least 1.000 battle-related deaths in on calendar year. \\
% Talk about variation in data and place map here?

The panel summary statistics for these variables are given in Figure \ref{tab:xtsum}. (Provide futher explanation about the data)

\subsection{Data Preparation for Model}
The authors clean and prepare their data before estimation. Some of these techniques we agree with, and others we have some theoretical issues with. The pros and cons of their methods will be discussed in further detail after the initial replication section.

\begin{itemize}
    \item Observations with missing values in the topic shares are filled forward. If $\theta_{it}$ is missing, and $\theta_{it - 1}$ is not missing, then $\theta_{it} <- \theta_{it - 1}.$
    \item The chosen conflict variable itself is not used as the dependent variable. The authors specifically look at two scenarios, either the onset or the incidence of conflict.
        \begin{itemize}
            \item Onset of conflict is defined as $Conflict_{t} = 0$ and $Conflict_{t + 1} = 1$. After creating this onset variable, all observations where $Conflict_{t} = 1$ are removed.
            \item Incidence of Conflict is defined as $Conflict_{t} = 1$ and $Conflict_{t + 1} = 1$. After creating this incidence variable, missing conflict observations are removed.
            \item In our replication, we will narrow our focus to only the onset of conflict as the authors define it.
        \end{itemize}
    \item Observations where the average population over the entire sample is less than 1000, and where population data is missing are removed.
    \item Observations where there are zero words written, or where this data is missing, are removed.
    \item As a robustness check, the authors provide the option to restrict the sample to only countries who have experienced conflict at least once in the entire sample.
\end{itemize}

\section{Model}
The aim of the model is to create forecasts for an armed conflict/ civil war outbreak in period $T+1$ at period $T \in \{1995,..., 2013\}$. To create this forecast, the full information set up to period $T$ is included into the forecast. Therefore, the respective country-year topic shares $\theta_{n,i,T}$ are calculated for every newspaper sub-sample available up to period $T$\footnote{As the amount of available articles/ words expands in $T$, the basis for defining a topic through characteristic words in $T$ does also expand. Hence, the every topic characteristic and every topic distribution will vary at every $T$} for each country $i$ and topic $n$. As a consequence, the following two steps are repeated at every $T$:

\noindent\textbf{Step 1: Estimate model and obtain fitted values}

\noindent From the model $y_{i,T+1} = \alpha + \beta_{i} + \theta_{i,T}\beta^{topics}$ the fitted values from the estimation based on the overall variation are obtained: 

\begin{equation}
    \hat{y}_{i,T+1}^{overall} = \hat{\alpha} + \hat{\beta_i} + \theta_{i,T}\hat{\beta}^{topics}
\end{equation}

\noindent From these fitted values that rely on the overall variation, the fitted fixed effects are subtracted in order to obtain the fitted within model:

\begin{equation}
    \hat{y}_{i,T+1}^{within} = \hat{\alpha} + \theta_{i,T}\hat{\beta}^{topics}
\end{equation}

\noindent \textbf{Step 2: Produce forecast based on fitted values for period T+1}

\noindent 1) The fitted values are transformed into binary variables depending on cutoff value c\\
2) Compare forecast (binary variable) to realizations of armed conflict and civil war\\
3) Assess performance of overall and within model by considering forecasting performance for any given value c through ROC curves 

\section{Replication Estimations}


\section{Extending the analysis}
We extend the authors' analysis by shifting the focus from a purely forecast driven evaluation to a more thorough understanding of the model.
This change of perspective ultimately aims at developing a model that is both better at truly assessing the underlying relationship between conflict
and events/topics (?) and better at enabling an intuitive interpretation of the resulting estimates.
We start by easing some of the restrictions the authors placed on the data, yielding a more balanced and complete panel data set.
We then continue with an assessment of the suitability of a set of panel data models, namely pooled OLS, fixed effects (the authors' model of choice)
and a dynamic panel data model, each in conjunction with the most suitable estimation strategy in this specific setting.
Lastly, we provide a thorough discussion of the possible sources of bias and outline mitigation strategies.

\subsection{Data Changes}

\subsection{Model Selection}


\subsection{Possible Sources of Bias}


\newpage

\begin{appendix}
    \section{Figures and Tables}
    \begin{longtable}{lllrrrr}

\toprule
               &        &  Mean &  Std. Dev. &    Min &   Max &  Observations \\
\textbf{Variable} & \textbf{Type} &       &            &        &       &               \\
\midrule
\multirow{3}{*}{\textbf{Armed Conflict}} & \textbf{overall} & 0.142 &      0.349 &  0.000 & 1.000 &          7520 \\
               & \textbf{between} &       &      0.020 &  0.106 & 0.186 &            40 \\
               & \textbf{within} &       &      0.349 & -0.044 & 1.036 &           188 \\
\cline{1-7}
\multirow{3}{*}{\textbf{Civil War}} & \textbf{overall} & 0.060 &      0.237 &  0.000 & 1.000 &          7520 \\
               & \textbf{between} &       &      0.024 &  0.027 & 0.112 &            40 \\
               & \textbf{within} &       &      0.236 & -0.052 & 1.033 &           188 \\
\cline{1-7}
\multirow{3}{*}{\textbf{Topic 1 Share}} & \textbf{overall} & 0.053 &      0.039 &  0.007 & 0.560 &          6639 \\
               & \textbf{between} &       &      0.005 &  0.046 & 0.063 &            39 \\
               & \textbf{within} &       &      0.038 & -0.002 & 0.561 &           185 \\
\cline{1-7}
\multirow{3}{*}{\textbf{Topic 2 Share}} & \textbf{overall} & 0.073 &      0.041 &  0.010 & 0.559 &          6639 \\
               & \textbf{between} &       &      0.010 &  0.050 & 0.089 &            39 \\
               & \textbf{within} &       &      0.040 &  0.004 & 0.549 &           185 \\
\cline{1-7}
\multirow{3}{*}{\textbf{Topic 3 Share}} & \textbf{overall} & 0.043 &      0.049 &  0.006 & 0.454 &          6639 \\
               & \textbf{between} &       &      0.003 &  0.038 & 0.051 &            39 \\
               & \textbf{within} &       &      0.049 &  0.004 & 0.451 &           185 \\
\cline{1-7}
\multirow{3}{*}{\textbf{Topic 4 Share}} & \textbf{overall} & 0.060 &      0.068 &  0.009 & 0.663 &          6639 \\
               & \textbf{between} &       &      0.012 &  0.032 & 0.080 &            39 \\
               & \textbf{within} &       &      0.067 & -0.006 & 0.663 &           185 \\
\cline{1-7}
\multirow{3}{*}{\textbf{Topic 5 Share}} & \textbf{overall} & 0.069 &      0.045 &  0.004 & 0.468 &          6639 \\
               & \textbf{between} &       &      0.008 &  0.045 & 0.081 &            39 \\
               & \textbf{within} &       &      0.045 & -0.003 & 0.476 &           185 \\
\cline{1-7}
\multirow{3}{*}{\textbf{Topic 6 Share}} & \textbf{overall} & 0.063 &      0.052 &  0.009 & 0.765 &          6639 \\
               & \textbf{between} &       &      0.011 &  0.036 & 0.081 &            39 \\
               & \textbf{within} &       &      0.051 & -0.003 & 0.774 &           185 \\
\cline{1-7}
\multirow{3}{*}{\textbf{Topic 7 Share}} & \textbf{overall} & 0.074 &      0.047 &  0.007 & 0.514 &          6639 \\
               & \textbf{between} &       &      0.006 &  0.063 & 0.086 &            39 \\
               & \textbf{within} &       &      0.046 & -0.005 & 0.509 &           185 \\
\cline{1-7}
\multirow{3}{*}{\textbf{Topic 8 Share}} & \textbf{overall} & 0.070 &      0.052 &  0.007 & 0.426 &          6639 \\
               & \textbf{between} &       &      0.006 &  0.058 & 0.084 &            39 \\
               & \textbf{within} &       &      0.051 & -0.006 & 0.420 &           185 \\
\cline{1-7}
\multirow{3}{*}{\textbf{Topic 9 Share}} & \textbf{overall} & 0.074 &      0.054 &  0.010 & 0.514 &          6639 \\
               & \textbf{between} &       &      0.012 &  0.058 & 0.116 &            39 \\
               & \textbf{within} &       &      0.053 & -0.024 & 0.519 &           185 \\
\cline{1-7}
\multirow{3}{*}{\textbf{Topic 10 Share}} & \textbf{overall} & 0.065 &      0.051 &  0.007 & 0.612 &          6639 \\
               & \textbf{between} &       &      0.008 &  0.053 & 0.092 &            39 \\
               & \textbf{within} &       &      0.051 & -0.009 & 0.605 &           185 \\
\cline{1-7}
\multirow{3}{*}{\textbf{Topic 11 Share}} & \textbf{overall} & 0.063 &      0.046 &  0.005 & 0.407 &          6639 \\
               & \textbf{between} &       &      0.010 &  0.047 & 0.082 &            39 \\
               & \textbf{within} &       &      0.044 & -0.008 & 0.410 &           185 \\
\cline{1-7}
\multirow{3}{*}{\textbf{Topic 12 Share}} & \textbf{overall} & 0.075 &      0.069 &  0.004 & 0.653 &          6639 \\
               & \textbf{between} &       &      0.017 &  0.058 & 0.135 &            39 \\
               & \textbf{within} &       &      0.067 & -0.044 & 0.654 &           185 \\
\cline{1-7}
\multirow{3}{*}{\textbf{Topic 13 Share}} & \textbf{overall} & 0.089 &      0.090 &  0.008 & 0.623 &          6639 \\
               & \textbf{between} &       &      0.010 &  0.070 & 0.103 &            39 \\
               & \textbf{within} &       &      0.090 & -0.001 & 0.614 &           185 \\
\cline{1-7}
\multirow{3}{*}{\textbf{Topic 14 Share}} & \textbf{overall} & 0.067 &      0.048 &  0.007 & 0.582 &          6639 \\
               & \textbf{between} &       &      0.005 &  0.058 & 0.076 &            39 \\
               & \textbf{within} &       &      0.048 &  0.006 & 0.579 &           185 \\
\cline{1-7}
\multirow{3}{*}{\textbf{Topic 15 Share}} & \textbf{overall} & 0.061 &      0.055 &  0.006 & 0.437 &          6639 \\
               & \textbf{between} &       &      0.007 &  0.048 & 0.075 &            39 \\
               & \textbf{within} &       &      0.055 & -0.006 & 0.429 &           185 \\
\bottomrule
\caption{Panel Data Summary}
\label{tab:xtsum}
\end{longtable} 

    % \begin{figure}[!htbp]
    %     \caption{Authors' Model Hypothesis}
    %     \centering
    %     \tikzstyle{box} = [rectangle, rounded corners, minimum width=1cm, minimum height=1cm,text centered, draw=black]
\tikzstyle{arrow} = [thick,->,>=stealth]

\begin{tikzpicture}[node distance = 2cm]
    \node (Yt+1) [box, fill = yellow!80!black, text width = 2cm] {\small $Conflict_{t + 1}$};

    \node (Xt) [box, left of = Yt+1, node distance = 4cm, fill = yellow!80!black, text width = 1.75cm] {\small $\textbf{Theta}_{t}$};

    \node (alpha) [box, below right = 1cm and 0.5cm of Xt, fill = yellow!80!black, text width = 0.75cm] {\small $\alpha_{i}$};
    \node (epsilon) [box, right of = Yt+1, node distance = 2.5cm, fill = yellow!80!black, text width = 0.75cm] {\small $\epsilon_{it}$};

    \draw [arrow] (Xt) -- (Yt+1);

    \draw[<->, bend right, thick] (alpha.east) to (Yt+1.south);
    \draw[<->, bend left, thick] (alpha.west) to (Xt.south);

    \draw[<->, thick] (epsilon.west) to (Yt+1.east);
\end{tikzpicture}
    %     \label{InitMod}
    % \end{figure}

    \begin{figure}[!htbp]
        \caption{Path Diagram of Model Hypothesis}
        \centering
        \tikzstyle{box} = [rectangle, rounded corners, minimum width=1cm, minimum height=1cm,text centered, draw=black]
\tikzstyle{arrow} = [thick,->,>=stealth]

\begin{tikzpicture}[node distance = 2cm]
    \node (Yt) [box, fill = yellow!80!black, text width = 2cm] {\small $Conflict_{t}$};
    \node (Yt+1) [box, right of = Yt, node distance = 3.5cm, fill = yellow!80!black, text width = 2cm] {\small $Conflict_{t + 1}$};
    \node (Yt+2) [box, right of = Yt+1, node distance = 3.5cm, fill = yellow!80!black, text width = 2cm] {\small $Conflict_{t + 2}$};
    \node (YT) [box, right of = Yt+2, node distance = 3.5cm, fill = yellow!80!black, text width = 2cm] {\small $Conflict_{T}$};

    \node (Et) [box, below of = Yt, node distance = 4cm, fill = yellow!80!black, text width = 2cm] {\small $Events_{t}$};
    \node (Et+1) [box, right of = Et, node distance = 3.5cm, fill = yellow!80!black, text width = 2cm] {\small $Events_{t + 1}$};
    \node (Et+2) [box, right of = Et+1, node distance = 3.5cm, fill = yellow!80!black, text width = 2cm] {\small $Events_{t + 2}$};
    \node (ET) [box, right of = Et+2, node distance = 3.5cm, fill = yellow!80!black, text width = 2cm] {\small $Events_{T}$};

    \node (Xt) [box, below of = Et, node distance = 2.5cm, fill = yellow!80!black, text width = 2cm] {\small $\textbf{Theta}_{t}$};
    \node (Xt+1) [box, right of = Xt, node distance = 3.5cm, fill = yellow!80!black, text width = 2cm] {\small $\textbf{Theta}_{t + 1}$};
    \node (Xt+2) [box, right of = Xt+1, node distance = 3.5cm, fill = yellow!80!black, text width = 2cm] {\small $\textbf{Theta}_{t + 2}$};
    \node (XT) [box, right of = Xt+2, node distance = 3.5cm, fill = yellow!80!black, text width = 2cm] {\small $\textbf{Theta}_{T}$};
    
    \node (X) [fit = (Xt) (XT), draw, black, dotted] {};
    \node (Y) [fit = (Yt) (YT), draw, black, dotted] {};
    \node (E) [fit = (Et) (ET), draw, black, dotted] {};

    \node (alpha) [box, below left = 1cm and 0.5cm of Yt, fill = yellow!80!black, text width = 0.75cm] {\small $\alpha_{i}$};
    \node (epsilon) [box, right of = YT, node distance = 3cm, fill = yellow!80!black, text width = 0.75cm] {\small $\epsilon_{it}$};

    \draw [arrow] (Yt) -- (Yt+1);
    \draw [arrow] (Yt+1) -- (Yt+2);
    \draw [arrow] (Yt+2) -- (YT);

    \draw [arrow] (Et) -- (Yt+1);
    \draw [arrow] (Et+1) -- (Yt+2);
    \draw [arrow] (Et+2) -- (YT);

    \draw [arrow] (Yt) -- (Et+1);
    \draw [arrow] (Yt+1) -- (Et+2);
    \draw [arrow] (Yt+2) -- (ET);

    \draw [arrow] (Et) -- (Et+1);
    \draw [arrow] (Et+1) -- (Et+2);
    \draw [arrow] (Et+2) -- (ET);

    \draw [arrow, <->, blue] (Et) -- (Yt);
    \draw [arrow, <->, blue] (Et+1) -- (Yt+1);
    \draw [arrow, <->, blue] (Et+2) -- (Yt+2);
    \draw [arrow, <->, blue] (ET) -- (YT);

    \draw [arrow, <-, green!45!black] (Xt) -- (Et);
    \draw [arrow, <-, green!45!black] (Xt+1) -- (Et+1);
    \draw [arrow, <-, green!45!black] (Xt+2) -- (Et+2);
    \draw [arrow, <-, green!45!black] (XT) -- (ET);

    \draw [arrow, <-, green!45!black, dotted] (Xt+1) -- (Et);
    \draw [arrow, <-, green!45!black, dotted] (Xt+2) -- (Et+1);
    \draw [arrow, <-, green!45!black, dotted] (XT) -- (Et+2);

    \draw[<->, bend left, thick] (alpha.north) to (Y.west);
    \draw[<->, bend right, thick] (alpha.south) to (E.west);

    \draw[<->, thick] (epsilon.west) to (Y.east);

    % \draw[arrow] (IV) -- (log_items) node[midway, fill = white, inner sep = 0pt] {IV};
    % \draw[arrow, <->] (unobs) -- (log_items);
\end{tikzpicture}
        \label{path}
    \end{figure}
\end{appendix}

\newpage

\section*{References}
Besley, Timothy and Torsten Persson. 2011. Pillars of prosperity: The political economics \newline \indent of development clusters. Princeton University Press.\vspace{0.25cm}

\noindent Chadefaux, Thomas. 2014. "Early warning signals for war in the news." Journal of Peace \newline \indent Research 51(1):5-18.\vspace{0.25cm}

\noindent Goldstone, Jack A, Robert H Bates, David L Epstein, Ted Robert Gurr, Michael B Lustik, \newline \indent Monty G Marshall, Jay Ulfelder and Mark Woodward. 2010. "A global model for \newline \indent forecasting political instability." American Journal of Political Science 54(1):190-208.\vspace{0.25cm}

\noindent Miguel, Edward and Shanker Satyanath. 2011. "Re-examining economic shocks and civil \newline \indent conflict." American Economic Journal: Applied Economics 3(4):228-232.\vspace{0.25cm}

\noindent Mueller, H., \& Rauh, C. (2018). "Reading Between the Lines: Prediction of Political \newline \indent Violence Using Newspaper Text." American Political Science Review, 112(2), 358-375. \newline \indent doi:10.1017/S0003055417000570\vspace{0.25cm}

\noindent Ward, Michael D, Nils W Metternich, Cassy L Dorff, Max Gallop, Florian M Hollenbach, \newline \indent Anna Schultz and Simon Weschle. 2013. "Learning from the past and stepping into \newline \indent the future: Toward a new generation of conflict prediction." International Studies \newline \indent Review 15(4):473-490.

\clearpage

\end{document}